\documentclass[a4paper,11p]{memoir}

\pdfoutput=1

\newlength{\figwidth}
% FINAL
%\setlength{\figwidth}{88mm}
% DRAFT
\setlength{\figwidth}{0.65\textwidth}


\usepackage{color}
\definecolor{links}{rgb}{0.7,0,0}   % red
\definecolor{urls}{rgb}{0,0,0.8}    % blue
\definecolor{cites}{rgb}{0,0,0.8}   % blue

\usepackage[colorlinks,hyperindex,linkcolor=links,citecolor=cites,urlcolor=urls]{hyperref} % generates colored links in pdf file
 
%---------------------------------------------------------
% styles and other macros
\usepackage[nosort]{cite}        
\usepackage{url} 
\usepackage[intlimits]{amsmath}
\usepackage{bbm}
\usepackage{graphicx}
\usepackage{paralist}
\usepackage{verbatim}
\usepackage{cprotect} %For using \cprotect\section{\verb|aaa|}
\usepackage[stretch=16,shrink=16,step=4]{microtype}
% \usepackage{vmr-symbols-vecbold}
% \usepackage{standard-macros}


\def\matn{\mathcal{N}}

%*******************************************************************************
%!TEX encoding =  

\begin{document}


\title{SPeCTrE\\
Short Packet Communication Toolbox for Wireless Engineers\\[1cm]
Version 0.0}

\maketitle

\begin{abstract}
  
\end{abstract}
\newpage
\tableofcontents

\newpage
%%%%%%%%%%%%%%%%%%%%%%%%%
\chapter{Motivation}
%%%%%%%%%%%%%%%%%%%%%%%%%
\chapter{AWGN channel}


General info:
\begin{itemize}
\item Maintainer: Y. Polyanskiy \url{<yp@mit.edu>}

\item Main references: \cite{PPV08,PPV10eneff}

\item Example: See Fig.~\ref{fig:awgn_example} 

\item Channel model:
	$$ Y_j = X_j + Z_j, \quad j=1,\ldots,n $$
	Noise: $Z_j \sim \matn(0,1)$ iid.\\
	Power constraints: Each codeword $x^n$ satisfies
			$$ \sum_{j=1}^n |x_j|^2 \le n P $$

\item Common input/output arguments:
\begin{enumerate}
\item \verb|P| -- input argument; parameter of the power constraint (so $P=10$ is $SNR=10~dB$).
\item \verb|epsil| -- block probability of block error.
\item \verb|lm| or \verb|Lms| -- output argument; $\log_2 M$, log-size of codebook (base-2 information units). 
\item \verb|n| -- input argument; blocklength.\\
		For slow functions that do not support vectorized arguments.
\item \verb|Ns| -- input argument; vector of blocklengths.
\end{enumerate}
\end{itemize}


\begin{figure}
\caption{Code and resulting picture for AWGN bounds}\label{fig:awgn_example}
\begin{verbatim}
	> plot_v3(....)
\end{verbatim}
\end{figure}

%\def\fname{\bgroup\cat

\cprotect\section{Summary plot: \verb|plot_v3()|}

Definition:
\begin{verbatim}
	function [Ns lb ub feinst gal wlfz] = plot_v3(P, epsil)
\end{verbatim}

This function plots various bounds on the same figure. See Fig.~\ref{fig:awgn_example}.


\cprotect\section{Achievability: \verb|shannon_ach2()|}

Definition:
\begin{verbatim}
	function lm = shannon_ach2(n, epsil, P)
\end{verbatim}

Function computes Shannon's cone-packing achievability bound, see~\cite[(41)]{PPV08}.

\cprotect\section{Achievability: \verb|kappabeta_ach()|}

\cprotect\section{Converse: \verb|betaq_low_v2()|}

\cprotect\section{Converse: \verb|wolfowitz()|}

\cprotect\section{Normal approximation: \verb|normapx_awgn()|, \verb|normapx_biawgn()|}

Format:
\begin{verbatim}
	function Lms = normapx_awgn(Ns, epsil, P);
\end{verbatim}

Fast and frequently very precise approximation to both achievability and converse:
	$$ \log M \approx n C - \sqrt{nV} Q^{-1}(\epsilon) + {1\over2} \log n $$


\cprotect\section{Code database: \verb|plot_universe()|}


\cprotect\section{Zero-rate (aka energy per bit, aka wideband regime)}
\textbf{TODO.}

%%%%%%%%%%%%%%%%%%%%%%%%%
\chapter{Fading channels}
%%%%%%%%%%%%%%%%%%%%%%%%%%%%
\bibliographystyle{IEEEtran}
\bibliography{IEEEabrv,refs}
\end{document}

